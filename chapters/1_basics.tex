\epigraph{

Excerpt from the play Rossum’s Universal Robots (RUR).\break
In the introductory scene Helena Glory is visiting Harry Domin the director general of Rossum’s Universal Robots and his robotic secretary Sulla.\break
\break
\textbf{Domin:} Sulla, let Miss Glory have a look at you.\break
\textbf{Helena (stands and offers her hand):} Pleased to meet you. It must be very hard for you out here, cut off from the rest of the world [the factory is on an island].\break
\textbf{Sulla:} I do not know the rest of the world Miss Glory. Please sit down.\break
\textbf{Helena (sits):} Where are you from?\break
\textbf{Sulla:} From here, the factory\break
\textbf{Helena:} Oh, you were born here.\break
\textbf{Sulla:} Yes I was made here.\break
\textbf{Helena (startled):} What?\break
\textbf{Domin (laughing):} Sulla isn’t a person, Miss Glory, she’s a robot.\break
\textbf{Helena:} Oh, please forgive me …
}{\textit{Karel Čapek \\ Rossum’s Universal Robots (R.U.R.)}}

\section{Classification of industrial robots and their structures}

Industrial robots can be classified according to various criteria: the number of degrees of freedom, kinematic structure, drives used, workspace geometry, motion characteristics, control method or programming method. According to the abovementioned criteria, several types of robots are distinguished:

\subsection*{Number of degrees of freedom}

\begin{itemize}
    \item Universal robot - 6 degrees of freedom
    \item Redundant robot - more than 6 degrees of freedom
    \item Deficient robot - less than 6 degrees of freedom
\end{itemize}

\subsection*{Kinematic structure}

\begin{itemize}
    \item Serial robots - with an open-loop kinematic chain
    \item Parallel robots - with a closed-loop kinematic chain
    \item Hybrid robots - combining both types of kinematic chains
\end{itemize}


\subsection*{Type of drives}

\begin{itemize}
    \item Electric
    \item Hydraulic
    \item Pneumatic
\end{itemize}

Currently, industrial robots with electric drives predominate in numbers. If high loads are required, hydraulic drives are used and for high speeds pneumatic drives are preferred.

\subsection*{Workspace geometry}

\begin{itemize}
    \item Cartesian - robots equipped with three perpendicular linear axes
    \item Cylindrical - robots equipped with two linear axes and one rotary axis
    \item Spherical -  robots equipped with two rotary axes and one linear axis
    \item Angular - robots equipped with a varying number of rotary axes
    \item SCARA - Selective Compliant Articulated Robot Arm, also equipped with with two rotary axes and one linear axis but in a different configuration than a spherical robotic arm
\end{itemize}

\section{Industrial robot programming}

Before we dive into the programming of industrial robotic arms, let us revise a few concepts.
A workcell represents a robot and a collection of machines or peripherals. A single robot controller is responsible for controlling the various appliances of a workcell.

A robot end effector is a peripheral placed at the end of the robotic arm. The end effector represents the last link of the robot. According to the application, end effectors can be grippers, welding devices, spray guns or grinding and deburring machines.

The robot controller is equipped with a so-called interface software. The interface software makes it easier for the user to program the robotic arm.  
Two basic information need to be programmed into the robotic arm:

\begin{itemize}
    \item position data - i.e. where to robotic arm has to move
    \item procedure - i.e. what action the robotic arm will perform at the specific position
\end{itemize}

Many different ways of teaching a robot position do exist:

\begin{itemize}
    \item Positional commands - the programmer inputs the position data using a GUI or a text based command
    \item Teach pendant - the Teach Pendant is a handheld control and programming unit attached to the robot controller
    \item Lead-by-the-nose - the robotic arm is de-energized and moved along the required positions by hand while the robotic controller saves them into memory
    \item Offline programming - the robotic cell including the robotic arm and machines are represented in a specialized software. The robot can be moved in the software and brand-specific applications for the robotic controller can be created.  

\end{itemize}

\subsection{Offline programming}
Different robot brands have incompatible interfaces. Robot programming languages evolve slowly, and robot manufacturers offer backwards compatibility. For example, FANUC ...

Offline programming  (OLP) refers to a programming method in which the robot is programmed outside the production environment. OLP helps to eliminate production downtime, allows studying multiple scenarios of a robot cell before setting up the productions cell. OLP also aids in predicting mistakes made in designing a work cell. The OLP workflow can be summarized into the following steps:

\begin{enumerate}
  \item Creating/importing a robotic cell that mimics the real life workcell. 
  \item Importing a CAD file representing the machined part
  \item Programming (graphical or textual) and automatic generation of robot path
  \item Simulation and validation - collision detection, join violations etc.
  \item Reporting - efficiency, productivity etc.
  \item Translating the robot program for a specific controller 
  \item Executing the robot program in a real workcell
\end{enumerate}
